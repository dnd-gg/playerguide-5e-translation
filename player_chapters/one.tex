\chapter{Einführung}
Das Dungeons and Dragons-Rollenspiel handelt von Storytelling in der Welt von Rittern und Zauberwesen. Du erinnerst dich bestimmt noch an die Zeit in der du als Kind in deine eigene Fantasiewelt abgetaucht bist und mit deinen Freunden Ritter oder Zauberer gespielt hast. Wie auch damals wirst du in  DaD von Phantasie gebrauch machen. Stell dir vor du stehst in einem stürmischen Nachthimmel vor einem Schloss umwoben mit Ranken. Eine alte Dame steht vor dem Eingang des Schlosses. In diesem Spiel geht es darum auf solche Situationen zu reagieren.

\begin{quotebox}
\textbf{Dungeon Master (DM):}
Nachdem ihr die kahlen Gipfel der Berge durchquert habt, biegt die Straße plötzlich nach Osten ab und Burg Rabenfels ragt vor euch auf. Alte und morsche Türme aus Stein bewachen stumm das Tor. Dahinter klafft ein breiter Abgrund hervor, der im tiefen Nebel verschwindet. Eine abgesenkte Zugbrücke überspannt den Abgrund und führt zu einem gewölbten Eingang zum Schlosshof. Die Ketten der Zugbrücke quietschen im Wind. Von den hohen, steinigen mit Moos bewachsenen Mauern aus werdet ihr von steinerne Gargoyles mit hohlen Augenhöhlen beobachtet. Jeder von ihnen grinst euch mit seinen hohlen Augenhöhlen abscheulich an.  Die Haupttore von Burg Rabenfels stehen offen.\\
\textbf{Phillip (spielt Gareth):} Ich habe das Gefühl mit den Gargoyles stimmt was nicht. Sie sehen viel zu echt aus. Wir sollten sie unter die Lupe nehmen.\\
\textbf{Amy (playing Riva):} Ich würde ersteinmal schauen wie robust sie ist, schließlich sieht sie sehr alt aus. Sie könnte bei unserem Gewicht zusammenbrechen.
\end{quotebox}

Im Gegensatz zu den Fantasiespielen als Kind, gibt DaD den Geschichten durch Struktur eine Möglichkeit, die Folgen der Aktion der Abenteurer zu bestimmen. Ein Spieler würfelt, um zu entscheiden, ob ihre Angriffe treffen oder verfehlen werden oder ob ihr Charakter der eine Klippe erklimmt, vor dem magischen Blitzschlag getroffen wird oder eine gefährliche Aufgabe bestehen muss. Alles ist möglich, aber die Würfel machen einige Ergebnisse wahrscheinlicher als andere.

\begin{quotebox}
\textbf{DM:} OK, einer nach dem anderen. Phillip, siehst du die Gargoyle an?\\
\textbf{Phillip:}
Ja. Gibt es einen Hinweis, dass es sich bei diesen Gestalten um Monster und nicht nur um Dekoration handeln könnte?\\
\textbf{DM:}
Mach einen Intelligenztest.\\
\textbf{Phillip:}
Trifft meine Untersuchungsfähigkeit zu?\\
\textbf{DM:}
Sicher!\\
\textbf{Phillip:} \textit{(ein d20 rollend)}
Ugh. Sieben.\\
\textbf{DM:}
Für dich sehen sie weiter wie harmlose Dekoration aus. Und Amy, Riva? Ihr überprüft die Zugbrücke?
\end{quotebox}

In Dungeons and Dragons erstellt jeder Spieler einen Abenteurer (auch Charakter genannt) und verbündet sich mit anderen Abenteurern (gespielt von Freunden). Gemeinsam könnte die Gruppe ein dunkles Verlies, eine zerstörte Stadt, eine verwunschene Burg, einen verlorenen Tempel im Dschungel oder eine mit Lava gefüllte Höhle unter einem mysteriösen Berg erkunden. Die Abenteurer können Rätsel lösen, sich mit anderen Charakteren unterhalten, grauenvolle Monster bekämpfen und magische Gegenstände und andere Schätze entdecken. Ein Spieler übernimmt jedoch die Rolle des Dungeon Master (DM), des führenden Geschichtenerzählers und des Schiedsrichters des Spiels. Der DM erstellt Abenteuer für die Charaktere, die durch die Gefahren irren und Entscheidungen darüber treffen, welche Pfade zu erkunden sind. Der DM könnte den Eingang zu Burg Rabenfels beschreiben und die Spieler entscheiden, was ihre Abenteurer tun sollen. Gehen sie über die gefährlich verwitterte Zugbrücke? Binden Sie sich mit einem Seil zusammen, um die Wahrscheinlichkeit zu minimieren, dass jemand herunterfällt, wenn die Zugbrücke nachgibt. Oder beschwören sie einen Zauber, der sie über den Abgrund trägt?

Dann bestimmt der DM die Ergebnisse der Aktionen der Abenteurer und erzählt, was sie erleben. Da der DM improvisieren kann, um auf alles zu reagieren, was die Spieler versuchen, ist DaD unendlich flexibel und jedes Abenteuer kann aufregend und unerwartet sein. Das Spiel hat kein wirkliches Ende. Wenn eine Geschichte oder ein Quest abgeschlossen ist, kann eine andere beginnen und so kann man eine fortlaufende Geschichte erstellen, die als Kampagne bezeichnet wird. Viele Leute, die das Spiel spielen, halten ihre Kampagnen über Monate oder Jahre aufrecht und treffen sich jede Woche mit ihren Freunden, um die Geschichte dort weiter zu spielen, wo sie aufgehört haben. Die Abenteurer werden mit der Kampagne immer mächtiger. Jedes Monster, das besiegt wurde, jedes Abenteuer vollbracht und jeder gefundene Schatz trägt nicht nur zur weiteren Geschichte bei, sondern bringt den Abenteurern auch neue Fähigkeiten ein. Diese Steigerung der Macht spiegelt sich im Niveau eines Abenteurers wider. Im Dungeons & Dragons-Spiel gibt es kein Gewinnen und Verlieren - zumindest nicht so, wie diese Begriffe normalerweise verstanden werden. Der DM und die Spieler entwickeln gemeinsam eine aufregende Geschichte mutiger Abenteurer, die sich tödlichen Gefahren stellen. Manchmal kommt es zu einem grausamen Ende eines Abenteurers, der von grausamen Monstern zerrissen oder von einem ruchlosen Bösewicht getötet wird. Trotzdem können die anderen Abenteurer nach mächtiger Magie suchen, um ihren gefallenen Kameraden wiederzubeleben, oder der Spieler könnte sich dafür entscheiden, einen neuen Charakter zu kreieren, um weiterzumachen. Die Gruppe kann ein Abenteuer nicht erfolgreich abschließen, jedoch wenn alle einen schönen Abend oder Mittag hatten und eine unvergessliche Geschichte kreiert und ein magisches Abenteuer erlebt hatten, dann haben alle gewonnen.

\section{Welten des Abenteuers}
Die vielen Welten in Dungeons and Dragons sind Orte der Magie und der Monster, der tapferen Krieger und spektakulären Abenteuer. Sie beginnen mit einer Priese mittelalterlicher Phantasie und fügt dann die Kreaturen, Orte und Magie hinzu, die diese Welten einzigartig machen.
Die Welten von Dungeons and Dragons existieren in einem riesigen Kosmos, genannt \textbf{Multiversum}, der auf seltsame und geheimnisvolle Weise miteinander und mit anderen Ebenen der Existenz verbunden ist, wie die Elementarebene des Feuers und der unendlichen Tiefe des Abyss. In diesem Multiversum gibt es eine unendliche Vielfalt von Welten.
Viele von ihnen wurden als offizielle Settings für DaD veröffentlicht. Die Legenden der vergessenen Reiche, Drachenlanze, Graufalk, Schwarze Sonne, Mystara und Eberron sind im Netz des Multiversums miteinander verwoben. Neben diesen Welten gibt es Hunderttausende mehr, die von Generationen von DaD-Spielern für ihre eigenen Spiele geschaffen wurden. Und inmitten des Reichtums des Multiversums könnt ihr auch eure eigene Welt schaffen.
Alle diese Welten ist einzigartig und jede Welt ist durch ihre eigene Geschichte und Kultur, unverwechselbare Monster und Rassen, fantastische Geografien, uralte Dungeons und schurkische Schurken gekennzeichnet. Einige Rassen haben ungewöhnliche Eigenschaften in verschiedenen Welten. Die Halblinge der dunklen Sonne sind beispielsweise Kannibalen, die im Dschungel leben, und die Elfen sind Wüstennomaden. In einigen Welten gibt es Rassen, die in anderen Umgebungen unbekannt sind [...]. Einige Welten werden von einer großen Geschichte beherrscht, wie zum Beispiel der Krieg der Lanze, der eine zentrale Rolle in dem Drachenlanze Setting spielt. Aber es sind alles DaD-Welten, und Sie können die Regeln in diesem Buch verwenden, um einen Charakter zu erstellen und in einer von ihnen zu spielen. Ihr DM setzt die Kampagne möglicherweise auf eine dieser Welten oder auf eine, die er erstellt hat. Da die DaD-Welten so vielfältig sind, solltest du dich mit deinem DM über die Regeln erkundigen, die das Spiel beeinflussen. Letztendlich ist der Dungeon Master die Autorität für die Kampagne und es ist sein Setting, selbst wenn das Setting eine vorgefertigte Welt ist.
\newpage
\section{Wie verwendet man die Regeln?}
