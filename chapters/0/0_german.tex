\chapter{Einführung}
Das Dungeons and Dragons-Rollenspiel handelt von Storytelling in der Welt von Rittern und Zauberwesen. Du erinnerst dich bestimmt noch an die Zeit in der du als Kind in deine eigene Fantasiewelt abgetaucht bist und mit deinen Freunden Ritter oder Zauberer gespielt hast. Wie auch damals wirst du in  DnD von Phantasie gebrauch machen. Stell dir vor du stehst in einem stürmischen Nachthimmel vor einem Schloss umwoben mit Ranken. Eine alte Dame steht vor dem Eingang des Schlosses. In diesem Spiel geht es darum auf solche Situationen zu reagieren.

\begin{quotebox}
\textbf{Dungeon Master (DM):}
Nachdem ihr die kahlen Gipfel der Berge durchquert habt, biegt die Straße plötzlich nach Osten ab und Burg Rabenfels ragt vor euch auf. Alte und morsche Türme aus Stein bewachen stumm das Tor. Dahinter klafft ein breiter Abgrund hervor, der im tiefen Nebel verschwindet. Eine abgesenkte Zugbrücke überspannt den Abgrund und führt zu einem gewölbten Eingang zum Schlosshof. Die Ketten der Zugbrücke quietschen im Wind. Von den hohen, steinigen mit Moos bewachsenen Mauern aus werdet ihr von steinerne Gargoyles mit hohlen Augenhöhlen beobachtet. Jeder von ihnen grinst euch mit seinen hohlen Augenhöhlen abscheulich an.  Die Haupttore von Burg Rabenfels stehen offen.\\
\textbf{Phillip (spielt Gareth):} Ich habe das Gefühl mit den Gargoyles stimmt was nicht. Sie sehen viel zu echt aus. Wir sollten sie unter die Lupe nehmen.\\
\textbf{Amy (spielt Riva):} Ich würde ersteinmal schauen wie robust sie ist, schließlich sieht sie sehr alt aus. Sie könnte bei unserem Gewicht zusammenbrechen.
\end{quotebox}

Im Gegensatz zu den Fantasiespielen als Kind, gibt DnD den Geschichten durch Struktur eine Möglichkeit, die Folgen der Aktion der Abenteurer zu bestimmen. Ein Spieler würfelt, um zu entscheiden, ob ihre Angriffe treffen oder verfehlen werden oder ob ihr Charakter der eine Klippe erklimmt, vor dem magischen Blitzschlag getroffen wird oder eine gefährliche Aufgabe bestehen muss. Alles ist möglich, aber die Würfel machen einige Ergebnisse wahrscheinlicher als andere.

\begin{quotebox}
\textbf{DM:} OK, einer nach dem anderen. Phillip, siehst du die Gargoyle an?\\
\textbf{Phillip:}
Ja. Gibt es einen Hinweis, dass es sich bei diesen Gestalten um Monster und nicht nur um Dekoration handeln könnte?\\
\textbf{DM:}
Mach einen Intelligenztest.\\
\textbf{Phillip:}
Trifft meine Untersuchungsfähigkeit zu?\\
\textbf{DM:}
Sicher!\\
\textbf{Phillip:} \textit{(ein d20 rollend)}
Ugh. Sieben.\\
\textbf{DM:}
Für dich sehen sie weiter wie harmlose Dekoration aus. Und Amy, Riva? Ihr überprüft die Zugbrücke?
\end{quotebox}

In Dungeons and Dragons erstellt jeder Spieler einen Abenteurer (auch Charakter genannt) und verbündet sich mit anderen Abenteurern (gespielt von Freunden). Gemeinsam könnte die Gruppe ein dunkles Verlies, eine zerstörte Stadt, eine verwunschene Burg, einen verlorenen Tempel im Dschungel oder eine mit Lava gefüllte Höhle unter einem mysteriösen Berg erkunden. Die Abenteurer können Rätsel lösen, sich mit anderen Charakteren unterhalten, grauenvolle Monster bekämpfen und magische Gegenstände und andere Schätze entdecken. Ein Spieler übernimmt jedoch die Rolle des Dungeon Master (DM), des führenden Geschichtenerzählers und des Schiedsrichters des Spiels. Der DM erstellt Abenteuer für die Charaktere, die durch die Gefahren irren und Entscheidungen darüber treffen, welche Pfade zu erkunden sind. Der DM könnte den Eingang zu Burg Rabenfels beschreiben und die Spieler entscheiden, was ihre Abenteurer tun sollen. Gehen sie über die gefährlich verwitterte Zugbrücke? Binden Sie sich mit einem Seil zusammen, um die Wahrscheinlichkeit zu minimieren, dass jemand herunterfällt, wenn die Zugbrücke nachgibt. Oder beschwören sie einen Zauber, der sie über den Abgrund trägt?

Dann bestimmt der DM die Ergebnisse der Aktionen der Abenteurer und erzählt, was sie erleben. Da der DM improvisieren kann, um auf alles zu reagieren, was die Spieler versuchen, ist DnD unendlich flexibel und jedes Abenteuer kann aufregend und unerwartet sein. Das Spiel hat kein wirkliches Ende. Wenn eine Geschichte oder ein Quest abgeschlossen ist, kann eine andere beginnen und so kann man eine fortlaufende Geschichte erstellen, die als Kampagne bezeichnet wird. Viele Leute, die das Spiel spielen, halten ihre Kampagnen über Monate oder Jahre aufrecht und treffen sich jede Woche mit ihren Freunden, um die Geschichte dort weiter zu spielen, wo sie aufgehört haben. Die Abenteurer werden mit der Kampagne immer mächtiger. Jedes Monster, das besiegt wurde, jedes Abenteuer vollbracht und jeder gefundene Schatz trägt nicht nur zur weiteren Geschichte bei, sondern bringt den Abenteurern auch neue Fähigkeiten ein. Diese Steigerung der Macht spiegelt sich im Niveau eines Abenteurers wider. Im Dungeons & Dragons-Spiel gibt es kein Gewinnen und Verlieren - zumindest nicht so, wie diese Begriffe normalerweise verstanden werden. Der DM und die Spieler entwickeln gemeinsam eine aufregende Geschichte mutiger Abenteurer, die sich tödlichen Gefahren stellen. Manchmal kommt es zu einem grausamen Ende eines Abenteurers, der von grausamen Monstern zerrissen oder von einem ruchlosen Bösewicht getötet wird. Trotzdem können die anderen Abenteurer nach mächtiger Magie suchen, um ihren gefallenen Kameraden wiederzubeleben, oder der Spieler könnte sich dafür entscheiden, einen neuen Charakter zu kreieren, um weiterzumachen. Die Gruppe kann ein Abenteuer nicht erfolgreich abschließen, jedoch wenn alle einen schönen Abend oder Mittag hatten und eine unvergessliche Geschichte kreiert und ein magisches Abenteuer erlebt hatten, dann haben alle gewonnen.

\section{Welten des Abenteuers}
Die vielen Welten in Dungeons and Dragons sind Orte der Magie und der Monster, der tapferen Krieger und spektakulären Abenteuer. Sie beginnen mit einer Priese mittelalterlicher Phantasie und fügt dann die Kreaturen, Orte und Magie hinzu, die diese Welten einzigartig machen.
Die Welten von Dungeons and Dragons existieren in einem riesigen Kosmos, genannt \textbf{Multiversum}, der auf seltsame und geheimnisvolle Weise miteinander und mit anderen Ebenen der Existenz verbunden ist, wie die Elementarebene des Feuers und der unendlichen Tiefe des Abyss. In diesem Multiversum gibt es eine unendliche Vielfalt von Welten.
Viele von ihnen wurden als offizielle Settings für DnD veröffentlicht. Die Legenden der vergessenen Reiche, Drachenlanze, Graufalk, Schwarze Sonne, Mystara und Eberron sind im Netz des Multiversums miteinander verwoben. Neben diesen Welten gibt es Hunderttausende mehr, die von Generationen von DnD-Spielern für ihre eigenen Spiele geschaffen wurden. Und inmitten des Reichtums des Multiversums könnt ihr auch eure eigene Welt schaffen.
Alle diese Welten ist einzigartig und jede Welt ist durch ihre eigene Geschichte und Kultur, unverwechselbare Monster und Rassen, fantastische Geografien, uralte Dungeons und schurkische Schurken gekennzeichnet. Einige Rassen haben ungewöhnliche Eigenschaften in verschiedenen Welten. Die Halblinge der dunklen Sonne sind beispielsweise Kannibalen, die im Dschungel leben, und die Elfen sind Wüstennomaden. In einigen Welten gibt es Rassen, die in anderen Umgebungen unbekannt sind [...]. Einige Welten werden von einer großen Geschichte beherrscht, wie zum Beispiel der Krieg der Lanze, der eine zentrale Rolle in dem Drachenlanze Setting spielt. Aber es sind alles DnD-Welten, und Sie können die Regeln in diesem Buch verwenden, um einen Charakter zu erstellen und in einer von ihnen zu spielen. Ihr DM setzt die Kampagne möglicherweise auf eine dieser Welten oder auf eine, die er erstellt hat. Da die DnD-Welten so vielfältig sind, solltest du dich mit deinem DM über die Regeln erkundigen, die das Spiel beeinflussen. Letztendlich ist der Dungeon Master die Autorität für die Kampagne und es ist sein Setting, selbst wenn das Setting eine vorgefertigte Welt ist.
\newpage
\section{Verwenden dieser Regeln}
Dieses Dokument zu den DnD-Grundregeln besteht aus drei Teilen.\\
In \textbf{Teil 1} geht es darum, einen Charakter zu erstellen und dich mit den Regeln und Informationen zu versorgen, die du benötigst, um deinen Charakter zu erstellen. Dieser Teil enthält Informationen zu den verschiedenen Rassen, Klassen, Hintergründen, Ausrüstung und anderen Anpassungsoptionen, aus denen du wählen kannst. Viele der Regeln in Teil 1 sind Grundlagen für Teil 2 und 3.\\
In \textbf{Teil 2} werden die Spielregeln beschrieben, die über die in dieser Einführung beschriebenen Grundlagen hinausgehen. Dieser Teil deckt die Arten von Würfeln ab, mit denen du den Erfolg oder Misserfolg der Handlungen deines Charakters bestimmen kannst, und beschreibt die drei großen Kategorien von Aktivitäten im Spiel: Erkundung, Interaktion und Kampf.\\
In \textbf{Teil 3} dreht sich alles um Magie. Dieser Teil behandelt den Ursprung der Magie in den Welten von DnD, die Regeln für Zaubersprüche und eine Auswahl von typischen Zaubersprüchen, die für Charaktere (und Monster) mit Magie verfügbar sind.

\section{Spielanleitung}
Das Spiel eines Dungeons and Dragons-Spiels verläuft nach diesem Grundmuster:\\
\textbf{1. Der DM beschreibt die Umgebung.} Der DM teilt den Spielern mit, wo sich ihre Abenteurer befinden und was um sie herum ist, und zeigt die grundsätzlichen Möglichkeiten der sich bietenden Optionen auf (wie viele Türen führen aus einem Raum heraus, was steht auf einem Tisch, wer ist in der Taverne usw.).\\
\textbf{2. Die Spieler beschreiben, was sie wollen.}
Manchmal spricht ein Spieler auch für die gesamte Gruppe und sagt zum Beispiel: \"Wir nehmen die Osttür\". Andere Male tun verschiedene Charaktere verschiedene Dinge: Ein Abenteurer sucht nach einer Schatzkiste, ein anderer untersucht ein mystisches Symbol, das an einer Wand eingraviert ist, und ein dritter hält nach Monstern Ausschau. Die Spieler müssen sich nicht abwechseln, aber der DM hört jedem Spieler zu und entscheidet, wie diese Aktionen ausgeführt werden. Manchmal ist das Lösen einer Aufgabe einfach. Wenn ein Abenteurer durch ein Zimmer geht und eine Tür öffnen möchte, könnte der DM einfach sagen, dass sich die Tür öffnet und beschreibt, was dahinter liegt. Die Tür ist jedoch möglicherweise verriegelt, der Boden kann eine tödliche Falle verdecken oder ein anderer Umstand macht es für Abenteurer schwierig, eine Aufgabe zu erledigen. In diesen Fällen entscheidet der DM, was passiert, und stützt sich häufig auf das Werfen eines Würfels, um die Ergebnisse einer Aktion zu bestimmen.\\
\textbf{3. Die DM erzählt die Ergebnisse der Aktionen der Abenteurer.}
Die Beschreibung der Ergebnisse führt häufig zu einem weiteren Entscheidungsknoten, der den Spielfluss wieder auf Schritt 1 zurückführt. Dieses Muster gilt unabhängig davon, ob die Abenteurer vorsichtig eine Ruine erkunden, mit einem hinterhältigen Prinzen sprechen oder sich in einem tödlichen Kampf gegen einen mächtigen Drachen befinden. In bestimmten Situationen, insbesondere im Kampf, ist die Aktion strukturierter und die Spieler (und DM) wählen abwechselnd Aktionen aus und lösen diese. Meistens ist das Spiel jedoch flüssig und flexibel und passt sich den Umständen des Abenteuers an. Häufig findet die Action eines Abenteuers in der Vorstellung der Spieler und der DM statt, wobei die verbalen Beschreibungen der DM zum Einsatz kommen. Einige DMs verwenden gerne Musik, Bilder oder aufgenommene Soundeffekte, um die Stimmung zu bestimmen, und viele Spieler und DMs verwenden unterschiedliche Stimmen für die verschiedenen Abenteurer, Monster und anderen Charaktere, die sie im Spiel spielen. Manchmal kann ein DM eine Karte auslegen und mithilfe von Token oder Miniaturfiguren jede an einer Szene beteiligte Kreatur darstellen, um den Spielern zu helfen, den Überblick zu behalten, wo sich alle Charaktere aufhalten.
\newpage
\subsection{Würfel}
Das Spiel verwendet polyedrische Würfel mit unterschiedlicher Seitenanzahl. Würfel wie diese finden Sie in Game Stores und in vielen Buchhandlungen. In diesen Regeln werden die verschiedenen Würfel mit dem Buchstaben d gefolgt von der Anzahl der Seiten bezeichnet: d4, d6, d8, d10, d12 und d20. Zum Beispiel ist ein d6 ein sechsseitiger Würfel (der typische Würfel, den viele Spiele verwenden). Ein \"Percentile-Würfel\" oder d100 funktioniert etwas anders. Man würfelt eine Zahl zwischen 1 und 100, indem Sie zwei verschiedene zehnseitige Würfel mit der Nummer 0 bis 9 würfeln. Ein Würfel gibt die Zehnerstelle und der andere die Einsstelle an. Welcher die Zehner und welcher die Einerstellen angibt wird natürlich vor dem Würfeln festgelegt. \\
Wenn du beispielsweise eine 7 und eine 1 würfelst, lautet die gewürfelte Zahl 71. Zwei Nullen stehen für 100. Einige zehnseitige Würfel sind in Zehnerzahlen (00, 10, 20 usw.) nummeriert um die Zehnerstelle von der Einerstelle zu unterscheiden. In diesem Fall ist ein Wurf von 70 und 1 eine 71, und 00 und 0 ist 100. Wenn Sie würfeln müssen, sagen Ihnen die Regeln, wie viele Würfel von einem bestimmten Typ gewürfelt werden sollen und welche Modifikatoren hinzugefügt werden müssen. Zum Beispiel bedeutet „3d8 + 5“, dass Sie drei achteckige Würfel werfen, diese zusammen addieren und der Summe 5 hinzufügen. Die gleiche Schreibweise erscheint in den Ausdrücken „1d3“ und „1d2“. Um den Wurf von 1d3 zu simulieren, würfeln Sie ein d6 und teilen Sie die gewürfelte Zahl durch 2 (Aufrunden).\\
Um den Wurf von 1d2 zu simulieren, würfel einfach einen beliebigen Würfel und gebe dem Wurf eine 1 oder 2 als Wert, je nachdem, ob der gewürfelte Wert gerade oder ungerade ist. (Wenn die gewürfelte Anzahl mehr als die Hälfte der Anzahl der Seiten des Würfels ist, ist dies eine 2.)

\subsection{Der D20}
Verletzt ein Schwert eines Abenteurers einen Drachen oder springt das Schwert einfach an den eisenharten Schuppen entzwei? Wird der Oger deinem Ablenkungsmanöver reinfallen? Kann ein Charakter über einen tosenden Fluss schwimmen? Kann ein Charakter einen Critical Hit eines Feuerballs vermeiden, oder nimmt er oder sie den vollen Schaden von der Flamme? In Fällen, in denen das Ergebnis einer Aktion unsicher ist, verlässt sich das Dungeons and Dragons-Spiel auf einen 20-seitigen Würfel (d20), um den Erfolg oder Misserfolg zu bestimmen.\\
Jeder Charakter und jedes Monster im Spiel verfügt über Fähigkeiten, die durch sechs \textbf{Fähigkeitswerten} definiert werden. Die Fähigkeiten sind Stärke, Geschicklichkeit, Konstitution, Intelligenz, Weisheit und Charisma. Für die meisten Abenteurer reichen sie normalerweise von 3 bis 18. (Monster können Werte von 1 bis 30 aufweisen.) Diese Fähigkeitspunkte und die von ihnen abgeleiteten \textbf{Fähigkeitsmodifizierer} sind die Grundlage für fast jeden D20-Wurf, den ein Spieler im Namen eines Charakters oder Monsters macht. Fähigkeitsüberprüfungen, Angriffswürfe und Rettungswürfe sind die drei Hauptarten von D20-Würfeln, die den Kern der Spielregeln bilden. Alle drei folgen diesen einfachen Schritten.\\
\textbf{1. Wirf den Würfel und füge einen Modifikator hinzu.}
Wirf einen d20 und füge den entsprechenden Modifikator hinzu. Dies ist in der Regel der Modifikator, der von einem der sechs Fähigkeitswerten abgeleitet wird, und enthält manchmal einen Fähigkeitsbonus, um die besondere Fähigkeit eines Charakters widerzuspiegeln. (In Kapitel 1 findest du Details zu den einzelnen Fähigkeiten und zur Bestimmung des Modifikators einer Fähigkeit.)\\
\textbf{2. Wenden Sie Nebenboni und Strafen an.} Ein Klassenmerkmal, ein Zauber, ein bestimmter Umstand oder ein anderer Effekt kann den Punkten einen Bonus oder eine Strafe zufügen.\\
\textbf{3. Vergleiche die Summe mit dem Ziel} Wenn die Gesamtsumme der Zielanzahl entspricht oder diese übersteigt, ist die Fähigkeitsüberprüfung, der Angriffswurf oder der Rettungswurf ein Erfolg. Ansonsten ist es ein Fehlschlag. Der DM ist normalerweise derjenige, der die Zielanzahl bestimmt und den Spielern mitteilt, ob ihre Fähigkeitsprüfung, Angriffswürfe und Rettungswürfe erfolgreich sind oder fehlschlagen. Die Zielnummer für eine Fähigkeitsüberprüfung oder einen Rettungswurf wird als Schwierigkeitsklasse \textbf{(Difficulty Class: DC)} bezeichnet. Die Zielnummer für einen Angriffswurf wird als bezeichnet
Rüstungsklasse \textbf{(Armor Class: AC)}. Diese einfache Regel regelt die Lösung der meisten Aufgaben in DnD-Spielen. Kapitel 7 enthält detailliertere Regeln für die Verwendung des d20 im Spiel.
\subsection{Vorteile und Nachteile}

Manchmal werden eine Fähigkeitsüberprüfung, ein Angriffswurf oder ein Rettungswurf durch spezielle Situationen (Vorteil und Nachteil) modifiziert. Der Vorteil spiegelt die positiven Umstände einer d20-Rolle wider, während der Nachteil das Gegenteil widerspiegelt. Wenn du Vor- oder Nachteile hast, werfe beim Würfeln einen zweiten d20. Verwende die höhere der beiden Rollen, wenn du einen Vorteil hast, und verwenden die untere Rolle, wenn du einen Nachteil hast. Wenn du z. B. eine Benachteiligung hast und eine 17 und eine 5 würfelst, verwende die 5. Wenn du stattdessen einen Vorteil hast und diese Zahlen würfelst, verwenden Sie die 17. Detaillierte Regeln für Vor- und Nachteile findest du in Kapitel 7.

\subsection{Spezifische Beats Allgemein}
Dieses Buch enthält insbesondere in den Abschnitten 2 und 3 Regeln, die den Spielverlauf bestimmen. Viele Rassenmerkmale, Klassenmerkmale, Zaubersprüche, magische Gegenstände, Monsterfähigkeiten und andere Spielelemente verstoßen jedoch in gewisser Weise gegen die allgemeinen Regeln und bilden eine Ausnahme von der Funktionsweise des restlichen Spiels. Denken Sie daran: Wenn eine bestimmte Regel einer allgemeinen Regel widerspricht, gewinnt die bestimmte Regel. \\
Ausnahmen von den Regeln sind oft geringfügig. Zum Beispiel beherrschen viele Abenteurer keine Langbogen, aber jeder Waldelf tut dies aufgrund seiner Rasse. Diese Eigenschaft schafft eine geringfügige Ausnahme im Spiel. Andere Beispiele für das Brechen von Regeln sind auffälliger. Zum Beispiel kann ein Abenteurer normalerweise nicht durch Wände gehen, aber einige Zaubersprüche machen das möglich. Die meisten Ausnahmen von den Regeln entfallen bei Magie.
\subsection{Abrunden}
Es gibt eine weitere allgemeine Regel, die du zu Beginn kennen solltest. Immer, wenn du eine Zahl im Spiel teilst, rundest du sie ab, wenn du bei einem Bruch landest, auch wenn der Bruch die Hälfte oder mehr beträgt.

\section{Abenteuer}
Das Dungeons and Dragons-Spiel besteht aus einer Gruppe von Charakteren, die sich auf ein Abenteuer begeben, das ihnen der Dungeon-Meister präsentiert. Jeder Charakter bringt besondere Fähigkeiten in das Abenteuer, in Form von Punktzahlen und Fähigkeiten, Klassenmerkmalen, Rassenmerkmalen, Ausrüstung und magischen Gegenständen. Jeder Charakter ist anders, mit verschiedenen Stärken und Schwächen. Die beste Gruppe von Abenteurern ist eine, bei der sich die Charaktere ergänzen und die Schwächen ihrer Gefährten abdecken. Die Abenteurer müssen zusammenarbeiten, um das Abenteuer erfolgreich abzuschließen.\\
Das Abenteuer ist das Herzstück des Spiels, eine Geschichte mit einem Anfang, einer Mitte und einem Ende. Ein Abenteuer kann vom Dungeon-Meister erstellt oder aus dem Regal gekauft, angepasst und an die Bedürfnisse und Wünsche des DM angepasst werden. In jedem Fall bietet ein Abenteuer ein fantastisches Setting, sei es ein unterirdischer Dungeon, eine zerfallende Burg, ein Stück Wildnis oder eine blühende Stadt. Es verfügt über eine Vielzahl von Charakteren: Die Abenteurer, die von den anderen Spielern am Tisch entworfen und gespielt werden, sowie Nichtspieler-Charaktere (NPCs). Diese Charaktere könnten Gönner, Verbündete, Feinde, Mietlinge oder einfach nur Hintergrund-Statisten in einem Abenteuer sein. Oft ist einer der NPCs ein Bösewicht, dessen Handlungen die Action eines Abenteuers vorantreibt.\\
Die Charaktere werden im Verlauf ihrer Abenteuer mit verschiedenen Kreaturen, Objekten und Situationen konfrontiert, mit denen sie irgendwie umgehen müssen. Manchmal tun die Abenteurer und andere Kreaturen ihr Bestes, um sich im Kampf zu töten oder zu erbeuten. Zu anderen Zeiten sprechen die Abenteurer mit einem anderen Wesen (oder sogar einem magischen Objekt) mit einem bestimmten Ziel. Und oft verbringen die Abenteurer Zeit damit, ein Rätsel zu lösen, ein Hindernis zu umgehen, etwas zu finden, das versteckt ist, oder die aktuelle Situation aufzuklären. In der Zwischenzeit erkunden die Abenteurer die Welt und entscheiden, welchen Weg sie nehmen und was sie als Nächstes versuchen werden.\\
Abenteuer variieren in Länge und Komplexität. Ein kurzes Abenteuer kann nur einige Herausforderungen mit sich bringen, und es dauert nicht mehr als eine Spielesitzung. Ein langes Abenteuer kann hunderte von Gefechten, Interaktionen und anderen Herausforderungen beinhalten und Dutzende von Sessions zum Spielen benötigen, die sich über Wochen oder Monate in Echtzeit erstrecken. Das Ende eines Abenteuers ist in der Regel durch die Abenteurer gekennzeichnet, die sich in die Zivilisation begeben, um sich auszuruhen und die Beute ihrer Arbeit zu genießen.\\
Aber das ist nicht das Ende der Geschichte. Sie können sich ein Abenteuer als eine einzelne Episode einer Fernsehserie vorstellen, die aus mehreren aufregenden Szenen besteht. Eine Kampagne ist die gesamte Serie - eine Reihe von Abenteuern, die miteinander verbunden sind. Eine konsistente Gruppe Abenteurer folgt der Erzählung von Anfang bis Ende.

\subsection{Die drei Säulen des \\ Abenteuers}
Abenteurer können versuchen, alles zu tun, was sich ihre Spieler vorstellen können. Es kann jedoch hilfreich sein, über ihre Aktivitäten in drei großen Kategorien zu sprechen: Erkundung, soziale Interaktion und Kampfhandlungen.\\
\textbf{Erkundung} umfasst sowohl die Bewegung der Abenteurer durch die Welt als auch deren Interaktion mit Objekten und Situationen, die ihre Aufmerksamkeit erfordern. Erkundung ist das A und O in DnD, die beschreiben, was ihre Charaktere tun sollen, und der Dungeon-Meister sagt den Spielern, was als Ergebnis passiert. Im großen Stil könnten die Charaktere einen Tag damit verbringen, eine Ebene zu durchqueren, oder eine Stunde, um sich durch die unterirdischen Kavernen zu bewegen. Auf der kleinsten Kö könnte es bedeuten, dass ein Charakter einen Hebel in einem Dungeonraum zieht, um zu sehen, was passiert.
\textbf{Soziale Interaktion}
Die Abenteurer sprechen mit jemandem (oder etwas anderem). Es könnte bedeuten, dass ein gefangener Kundschafter den geheimen Eingang zum Goblin-Lager enthüllt, Informationen von einem geretteten Gefangenen erhält, von einem Ork-Häuptling um Gnade bittet oder einen gesprächigen Zauberspiegel überredet, den Abenteurern einen entfernten Ort zu zeigen.\\
Die Regeln in den Kapiteln 7 und 8 unterstützen Erkundung und soziale Interaktion, ebenso wie viele Klassenmerkmale in Kapitel 3 und Persönlichkeitsmerkmale in Kapitel 4.
\textbf{Kämpfe}, der Schwerpunkt von Kapitel 9, besteht darin, dass Charaktere und andere Kreaturen Waffen schwingen, Zaubersprüche aussprechen, Manövrierpositionen und vieles mehr ausführen - alles in dem Bemühen, ihre Gegner zu besiegen, egal ob es darum geht, jeden Feind oder Gefangene zu töten, oder einen Gegner zu besiegen. Der Kampf ist das strukturierteste Element einer DnD-Sitzung. Kreaturen wechseln sich ab, um sicherzustellen, dass jeder die Chance hat, zu handeln. Selbst im Rahmen einer Schlacht gibt es für Abenteurer immer noch die Möglichkeit, verrückte Stunts zu versuchen, beispielsweise eine Treppe auf einem Schild hinunter zu surfen, die Umgebung zu untersuchen (vielleicht durch einen mysteriösen Hebel zu ziehen) und mit anderen Kreaturen zu interagieren einschließlich Verbündeter, Feinde und neutrale Parteien.
\subsection{Die Wunder der Magie}
Nur wenige DnD-Abenteuer enden ohne Magie. Ob hilfreich oder schädlich: Magie taucht häufig im Leben eines Abenteurers auf und steht im Mittelpunkt der Kapitel 10 und 11.\\
In den Welten von Dungeons & Dragons sind Praktiker der Magie selten und werden durch ihr außergewöhnliches Talent von den Menschenmassen unterschieden. Gewöhnliche Menschen sehen regelmäßig Anzeichen von Magie, aber dies passiert normalerweise nicht sehr oft - ein Monster, ein sichtbar beantwortetes Gebet, ein Zauberer, der mit einem verzaubertem Schildwächter als Leibwächter durch die Straßen geht.\\
Für Abenteurer ist Magie jedoch der Schlüssel zu ihrem Überleben. Ohne die heilende Magie von Klerikern und Paladinen würden Abenteurer schnell ihren Wunden erliegen. Ohne die erhebende magische Unterstützung von Barden und Klerikern könnten Krieger von mächtigen Feinden überwältigt werden. Ohne die magische Kraft und Vielseitigkeit von Zauberern und Druiden würde sich jede Bedrohung verzehnfachen.\\
Magie ist auch ein bevorzugtes Werkzeug für das Böse. Viele Abenteuer werden durch die Machenschaften von Zauberkünstlern getrieben, die die Magie für die dunkle seite der Macht einsetzen. Ein Sektenführer versucht, einen Gott zu erwecken, der in den tiefen der Meere schlummert, eine Hexe entführt Jugendliche, um sie auf magische Weise zu entleeren, ein verrückter Zauberer, der eine Armee von Kriegern mithile eines Lebenszauberspruches erweckt, ein Drache beginnt ein mystisches Ritual als Gott der Zerstörung - dies sind nur einige der magischen Bedrohungen, denen Abenteurer gegenüberstehen. Mit ihrer eigenen Magie, in Form von Zaubersprüchen und magischen Gegenständen, können sich die Abenteurer durchsetzen!+
